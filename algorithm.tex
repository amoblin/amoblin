\documentclass[a4paper,12pt]{article}
\usepackage{amsmath}
\usepackage{leftidx}
\usepackage{listings}%插入C++源代码
\usepackage{hyperref}%插入超链接
%\usepackage{arydshin}
\usepackage{CJKutf8}
\title{算法中的映射思想}
\author{崔贵林}
\date{\today}
\begin{document}
\begin{CJK*}{UTF8}{gbsn}
	\maketitle
	\begin{abstract}
		本文通过两个例子阐述了算法中的映射思想,这种映射在将数据和一维数组下标之间建立起一一对应,在某些情况下会得到比较好的效果
	\end{abstract}
	\emph{Keywords}:算法,映射,一一对应,二项展开,组合数,序数。
	\newpage
	\section{打印图形}
	$\begin{array}{cccc}
		1\\
		5&2\\
		8&6&3\\
		10&9&7&4
	\end{array}$\\
	书本上主要使用层和数组之间的转化关系来解决问题。也就是最后那句话:综合以上分析,i层第j个数据对应的数组元素是 a[i-1+j][j]。\\
	这是一种方法,但缺点是占用了不少存储空间,并且将近一般没用上。太浪费了!\\
	能不能换一种方法?\\
	看下面的数:\\
	$\begin{array}{ccccccccccc}
		S_1:& 1& 2& 3& 4& 5& 6& 7& 8& 9& 10\\
		S_2:&(1,1)&(2,2)&(3,3)&(4,4)&(2,1)&(3,2)&(4,3)&(3,1)&(4,2)&(4,1)\\
		S_3:&(1,1)&(1,2)&(1,3)&(1,4)&(2,1)&(2,2)&(2,3)&(3,1)&(3,2)&(4,1)\\
	\end{array}$\\
	$S_1$是自然数列的子集,$S_2$中的元素是$S_1$中元素在二维数组下的坐标,$S_3$中元素是$S_1$中元素在对角线和垂线组成的斜交坐标系下的坐标。\\
	书本上的方法,说白了,就是寻找$S_2$和$S_3$的关系,然后从$S_3$构建$S_2$,最后打印输出$S_2$。\\
	而这里我要讲的方法,是从$S_1$直接到$S_2$。\\
	重新排列一下,把$S_2$放上面,按照$S_2$的先行后列顺序排列,于是得到:\\
	$\begin{array}{ccccccccccc}
		S_2^{'}:& (1,1)& (2,1)& (2,2)& (3,1)& (3,2)& (3,3)& (4,1)& (4,2)& (4,3)& (4,4)\\
		S_3^{'}:& (1,1)& (2,1)& (3,2)& (3,1)& (2,2)& (1,3)& (4,1)& (3,2)& (2,3)& (1,4)\\
		S_1^{'}:& 1& 5& 2& 8& 6& 3& 10& 9& 7& 4\\
	\end{array}$
	我们要做的,是直接由$S_2^{'}$找到$S_1^{'}$,然后打印输出,占用的空间几乎为零。\\
	但这个关系不是那么好找的。仔细看看。。。\\
	首先我们定义要输出的行数$n$。那么对于$S^{'}_2$中的元素$(i,j)$,$S_3^{'}$中对应为$(i+1-j,j)$.\\
	其次,$S_3^{'}$中元素$(i,j)$对应到$S_1^{'}$中元素$x$为$x = n + n-1 + n-2 + \ldots + n-(i-2) +j=(i-1) \times n - \frac{(i-1) \times (i-2)}{2} + j$\\
	那么 $S_2^{'}$中元素$(i,j)$对应到$S_1^{'}$中元素为$x = (i+1-j -1) \times n - (i+1-j-1) \times (i+1-j-2) \div 2 +j = (i-j)*n - (i-j)*(i-j-1)/2 +j$\\
	这样把$S_2^{'}$中下三角矩阵循环完毕,$S_1^{'}$中的元素也就计算,打印,输出完毕!\\
	具体的源代码请参阅http://code.google.com/p/calgorithm/source/browse/trunk/exercises.cc
	\section{二项展开和组合数的序数问题}
	例 从1,2,3,4四个数中\\
	$\begin{array}{ccccccccccccccccc}
		S_1:& 0& 1& 2& 3& 4& 5& 6& 7& 8& 9& 10& 11& 12& 13& 14& 15\\
		S_2:& 0& 1& 2& 3& 4& 12& 13& 14& 23& 24& 34& 123& 124& 134& 234& 1234\\
	\end{array}$
	我们知道二项展开和组合数是一一对应的。从上面也可以看出,两位数的个数是$C_4^2$,三位数的个数是$C_4^3$,四位数的个数是$C_4^4$。一位数的个数是$C_4^1$,这里0作为$C_4^0$了。\\
	那么我们要求$S_2$到$S_1$的映射函数$f:S_2 \rightarrow S_1$,该怎么办呢?\\
	对于$\forall x \in S_2, y \in S_2$,定义$d(x,y)=|f(x)-f(y)|$。\\
	比方,求$f(134)$,可以这样,先求$f(123)$,再求$d(123,134)$。而123是首个三位数,所以$f(123)$应该是一位数的个数加上两位数的个数再加一,即$f(123)=C_4^1+C_4^2+1=4+6+1=11$\\
	下面求$d(123,134)$。既然首位都是12,那么是不是可以去掉呢?就是求$d(23,34)$了。34是3开头的最小两位数,那么$d(23,34)=d(23,24)+1=d(3,4)+1=1+1=2$。\\
	所以$f(134)=11+2=13$\\
	这个例子的数太少,不具有一般性。下面我们用更多的数,使推广到更一般的情形。\\
	假设有$S_n=\{1,2,3,\cdots,n\}$,求$f$.\\
	为方便叙述,这里假设$n\ge 8$,现在求一下$f(2458)$。
	先讲两个概念:紧接前元和紧接后元。相邻的两个数,前者叫后者的紧接前元,后者叫前者的紧接后元。\\
	逻辑上是这样的,从末位开始,依次往前看。最后一位是8,前一位是5,8不是5的紧接后元,那么f(2458)=f(2456)+d(6,8)。\\
	下面求$f(2456)$。现在到了5的位置。5是4的紧接后元,那么继续往前看。4不是2的紧接后元,那么$f(2456)=f(2345)+d(2345,2456).$\\
	而$d(345,456)=d(345,3(n-1)n)+1=d(45,(n-1)n)+1=C_{n-3}^2+1$
	下面该2了,2是首位了,但不是最小的首位。那么$f(2345)=f(1234)+d(1234,2345)$
	$d(1234,2345)=d(1234,1(n-2)(n-1)n)+1=d(234,(n-2)(n-1)n)+1=C_{n-1}^3 + 1$\\
	$f(1234)=C_n^0+C_n^1+C_n^2+C_n^3$\\
	所以 $f(2458)=f(2456)+d(6,8)=f(2345)+d(2345,2456)+d(6,8)\\
	=f(2345)+d(345,456)+d(6,8)=f(1234)+d(1234,2345)+d(345,3(n-1)n)+1+d(6,8)\\
	=C_n^0+C_n^1+C_n^2+C_n^3+d(1234,1(n-2)(n-1)n)+1+d(45,(n-1)n)+1+d(6,8)\\
	=C_n^0+C_n^1+C_n^2+C_n^3+d(234,(n-2)(n-1)n)+1+C_{n-3}^2+1+d(6,8)\\
	=C_n^0+C_n^1+C_n^2+C_n^3+C_{n-1}^3+1+C_{n-3}^2+1+d(6,8)$\\
	即:
	$$f(2458)=\sum_{i=0}^{3}{C_n^i} + C_{n-1}^3 + 1 + C_{n-3}^2+1 + d(6,8) $$\\
	$$f(2458)=\sum_{i=0}^{3}{C_n^i} + \sum_{i=0}^{3} C_{n-1}^3 + 1 + C_{n-3}^2+1 + d(6,8) $$\\
	这是递归的算法。若要递推呢。是这样的。
	$f(2458)=f(1234)+d(1234,2345)+d(2345,2456)+d(2456,2458)\\
	=f(1234)+d(234,(n-2)(n-1)n)+1+d(45,(n-1)n)+1+d(6,8)$\\
	$$f(2458)=\sum_{i=0}^{3}{C_n^i} + C_{n-1}^3 + 1 + C_{n-3}^2+1 + d(6,8) $$\\
	问题提出:给定n,和不超过n长的数字串,求对应的序号。
	分析:使用一维数组存储该数字串,然后从前往后依次比较。
	核心算法:
	\lstset{language=C++}
	\lstset{breaklines}
	\begin{lstlisting}
		int order()
		{
			int order=0;
			for(int i=0;i<r;i++)
				order = order + Cnm(n,i);
			for(int j=1;j<r+1;j++)
			{
				for(int k=0;k<a[j]-a[j-1]-1;k++)
					order = order + Cnm(n-a[j-1]-1-k,r-j);
			}
			return order;
		}
	\end{lstlisting}
	详细算法参见\href{http://code.google.com/p/calgorithm/source/browse/trunk/comb.cc}{我的google code}中order()函数。
	这样任给一个组合数就可以得到它的序数。这在拓扑中由最简基拓扑生成拓扑的过程中大有用处。
	\begin{thebibliography}{99}
		\bibitem{book} 算法设计与分析,吕国英主编,清华大学出版社,2006年
	\end{thebibliography}
\end{CJK*}
\end{document}
