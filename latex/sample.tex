\documentclass[a4paper,12pt]{article}				%为注释符。导言区开始。模板声明,这里是论文版式
\usepackage{syntonly}								%debug用
\usepackage{verbatim}								%宏集,原样显示
\usepackage{CJKutf8}				%以下三段设置zh.CN_utf-8
\usepackage[unicode]{hyperref}      %书签,通常放在最后
\title{\LaTeX边学边用边总结}								%标题
\author{CSiP}								%作者
\date{\today}									%不写时自动加上系统时间
\begin{document}					%导言区结束,文档开始
\begin{CJK*}{UTF8}{gbsn}
	\maketitle								%显示标题,作者,时间
	\section{中文支持}
	\subsection{一首词的例子}
	\begin{center}
	\Large{满江红}\\
	\large岳飞\\
\end{center}
怒发冲冠,凭阑处、潇潇雨歇。抬望眼、仰天长啸,壮怀激烈。三十功名尘与土,八千里路云和月。莫等闲、白了少年头,空悲切。 \\							%\\为断行符
	靖康耻,犹未雪;臣子恨,何时灭。驾长车踏破贺兰山缺。壮志饥餐胡虏肉,笑谈渴饮匈奴血。待从头、收拾旧山河。朝天阙。
	\section{数学公式的显示}
	\texttt{Hello},\textrm{world}! {\LaTeX}				%特殊符号的引用
	can typeset equations like
	\begin{equation}					%开始
		\int^{2\pi}_0\sin^2\theta d\theta = \frac{1}{2}
		% \int是积分符,^后跟右上角标,_后跟右下角标,
	\end{equation}						%结束
	\begin{equation}					%开始
		\left[
		{\bf X} + {\rm a} \ \geq\
		\underline{\hat a} \sum_i^N \lim_{x \rightarrow k} \delta C
		\right]
	\end{equation}						%结束
	\subsection{喜欢 \LaTeX +pdf的理由}
	\begin{enumerate}
		\item 全球通用,就像xml一样
		\item 格式良好,长期有效
		\item 显示与平台无关
		\item 纯文本,可版本控制
	\end{enumerate}
	very cool!
\section{图片加载}
%\includegraphics{./csip_2005.png}
\end{CJK*}
\end{document}						%文档结束

