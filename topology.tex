\documentclass[a4paper,12pt]{article}
\usepackage{amsmath}
\usepackage{eucal}
\usepackage{listings}
\usepackage{CJKutf8}
\title{有限集拓扑的最简基研究}
\author{崔贵林}
\date{\today}
\begin{document}
\begin{CJK*}{UTF8}{gbsn}
	\maketitle
	\begin{abstract}
		本文主要研究在有限集中通过最简基生成拓扑的过程和其算法实现。
	\end{abstract}
	{keywords}:有限集,拓扑,最简基,算法
	\newpage
	\section{拓扑的基}
	定义:如果$X$是一个集合,$X$的某拓扑的一个基(basis)是$X$的子集的一个族$\mathcal{B}$(其成员称为基元素(basis element)),满足条件:\\
	(1) 对于每一个$x \in X$,至少存在一个包含$x$的基元素$B$.\\
	(2) 若$x$属于两个基元素$B_1$和$B_2$的交,则存在包含$x$的一个基元素$B_3$,使得$B_3 \subset B_1 \cap B_2$.\\
	由基生成拓扑的另一种方法由以下引理给出。\\
	引理:设X是一个集合,$\mathcal{B}$是X的拓扑$\tau$的一个基。则$\tau$等于$\mathcal{B}$中元素所有并的族。
	从拓扑基和基元素的定义里我们发现,没有对空集有特别要求,即空集可以是基元素,也可以不是基元素,无妨。但在引理中,说由基生成的拓扑等于基中元素所有并的族。如果这个并可以从零个并开始,并且说零个集合的并是空集的话,那么基中可以没有空集,否则基中必须有空集。否则空集由哪些基元素并出来呢?
	显然某一拓扑的拓扑基不唯一。
	\section{最简基}
	裴惠生在《河南大学学报》1987年第四期 “关于有限集的拓扑种类问题” 的论文中提出了最简基的概念。
	定义:设$X \ne \emptyset$,$W$为$X$上某一拓扑的基,若$\emptyset \notin W$,且$\forall B \in W$不存在$W^{'} \subset W - \{B\}$使$$B=\bigcup_{B^{'}\in W^{'}}{B^{'}}$$,则称$W$为$X$的一个最简基。
	最简基里明确了不包含空集。从这个最简基的定义里我们能够得出这样一个结论:令$\mathcal{W}$表示集合$X$的一拓扑$\tau$的所有拓扑基的全体。则有:\\$W$是$X$的拓扑$\tau$的最简基$$\Longleftrightarrow W=\bigcap_{W^{'}\in \mathcal{W}}{W^{'}}$$
	换句话说,最简基也就是最小的拓扑基。\\
	有了最简基的概念以后,拓扑和拓扑基之间就建立起一一对应关系了。\\
	裴惠生老师在论文里定义$X_n$为n点集。这里再定义所有n点集的全体为$\mathcal{X}_n$\\
	下面详细研究如何从最简基生成一个拓扑。\\
	\section{最简基生成拓扑}
	从最简拓扑基生成拓扑的步骤:\\
	\begin{enumerate}
		\item 空集$\emptyset$是开集,但不是最简基元素。
		\item $\forall x \in X_n$,判断单点集${x}$是否是开集?若${x}$是开集,则${x}$是最简基元素
		\item $\forall x\in X_n,y\in X_n$,分两种情况:$\alpha$,若${x}$和${y}$都是开集,则可以直接得出:1,${x,y}$是开集 2,${x,y}$不是最简基元素;$\beta$,若${x}$和${y}$至少有一个不是开集,则判断两点集{x,y}是否是开集?若{x,y}是开集,则{x,y}是最简基元素。记这样的两点集{x,y}的全体为集族$\mathcal{X}_2$。
		\item $\forall X\in \mathcal{X}_2,Y\in \mathcal{X}_2$,令$Z=X\cup Y$。分两种情况:$\alpha$,若X和Y都是开集,则可以直接得出:1,Z是开集。 2,Z不是最简基元素;$\beta$,若X和Y不全是开集,则判断Z是否是开集?若Z是开集,则Z是最简基元素。这里任意两个集合X和Y的并集Z可能有两种情况:1,$Z \in \mathcal{X}_3$,这时若Z第一次出现,则判断之,否则参照之前盼到你,而不必再判断。2,$Z \in \mathcal{X}_44$,这时不会重复出现,可直接判断。
		\item 仿照上述步骤重复下去,假设我们已得到了k元集的全体$\mathcal{X}_k$。对于$\forall X \in \mathcal{X}_k,Y\in \mathcal{X}_k$,令$Z=X\cup Y$,这时首先看看Z是否已经被确定为开集。因为在$k \ge 2$时就可以跨级生成了。即二元集和二元集不只生成三元集,还生成了四元集。事实上,这里k元集生成的集合的基数是k+1到2k,当然还有一个条件,就是不会超过全集的个数n。好,如果足够幸运,Z是第一次生成的,那么和前面一样,分两种情况判断,这里不再多说。并集Z的生成情况如前所述,集合基数的范围是$[k+1,min(2k,n)]\cap N.$
		\item 当判断到元素个数为n的时候,也就是全集$X_n$,这时有$\mathcal{X}_n={X_n}$。如果$X_n$早已被$\mathcal{X}_k$的某两个开元素并出来,那么$X_n$不是最简基元素,否则$X_n$是最简基元素。无论哪种情况,$X_n$是开集,结束。
	\end{enumerate}
	经过这样的步骤以后,一个拓扑,伴随着其最简基的确定,也就确定下来了。\\
	其实这个过程类似二项展开,首先是空集,其个数是$C_n^0$,然后是单点集,这样的集合有$C_n^1$个,由单点集并出来的二元集的全体和n元集的幂集中的二元集的全体是一样的,个数是$C_n^2$。后面的依次类推。\\
	\section{确定拓扑的最简基}
	对于一个确定的拓扑,如何找寻它的最简基呢?
	先看一般的拓扑基。\\
	引理:设$X$是一个拓扑空间,$\mathcal{C}$是$X$的开集的一个族,它满足对于X的每一个开集U以及每一个$x\in U$,存在$\mathcal{C}$的一个元素C,使得$x\in C\subset U$,那么$\mathcal{C}$就是X上这个拓扑的一个基。
	这是如何由给定的拓扑得到基的一种方法,但不具有可操作性,另外拓扑基也不唯一。下面研究如何由拓扑得到最简基。\\
	其实这就是对于拓扑中的每一个开集,判断它是否是最简基元素的过程。最简基元素区别于一般基元素或开集的本质属性在于什么呢?这是问题的关键。我们知道,开集可以是基元素,基元素可能是最简基元素。最简基元素一定是基元素,基元素一定是开集。一个开集,在特定的集合里,它就可以成为基元素,如果这个基元素又具备了某种性质,它就可以是最简基元素。基$\mathcal{C}$,最简基$\mathcal{W}$,拓扑$\tau$的关系是这样的:
	$$\mathcal{W} \subset \mathcal{C} \subset \tau$$
	如何判断一个开集是不是最简基元素?\\
	引理:设$(X,\tau)$是拓扑空间,W是开集,如果$\exists x \in W,$不$\exists B \in \tau$,使得$x\in B$,则W是最简基元素。\\
	显然。\\
	这个引理比较具有可操作性。可以使用递归判断。\\
	其实也可以使用上面最简基生成拓扑的过程来判断得到最简基。\\
	\section{编程实现}
	对于n点集$X_n$,使用长度为$2^n$的数组存储其拓扑结构。对于$\forall X\subset X_n$,X在数组中的存储位置为它的序号。这里定义一个映射$f:2^X \rightarrow N$\\
	下面程序初步实现了有限集的拓扑的子集的遍历,计算出每个子集的序号。\\
	下一步将设置三种状态:0 闭集 1 最简基元素 2 开集\\
	\lstset{language=C++}
	\lstset{breaklines}
	\begin{lstlisting}
class Topology
{
    public:
        Topology(int n)
        {
            this->n = n;
            int n2=1;
            for(int i=0;i<n;i++)
                n2=n2*2;
            a = new int[n2];
            for(int i=0;i<n2;i++)
                a[i]=0;
            b = new int[n];
            b[0]=0;
        }
        void init()
        {
            f(n);
        }
    protected:
        void f(int m)
        {
            if(m>1)
            {
                f(m-1);
            }
            g(1,m,m);
        }
        void g(int start,int r,int m)
        {
            if(r>0)
            {
                for(int i=start;i<n-r+2;i++)
                {
                    b[m-r+1]=i;
                    g(i+1,r-1,m);
                }
            }
            else
            {
                for(int i=1;i<m+1;i++)
                    cout<<b[i]<<" ";
                this->r = m;
                cout<<"order is "<<order()<<endl;
            }

        }
    protected:
        int order()
        {
            int order=0;
            for(int i=0;i<r;i++)
                order = order + Cnm(n,i);
            for(int j=1;j<r+1;j++)
            {
                for(int k=0;k<b[j]-b[j-1]-1;k++)
                    order = order + Cnm(n-b[j-1]-1-k,r-j);
            }
            return order;
        }
        int Cnm(int n,int m)
		{
            int cnm=1;
            for(int i=n;i>n-m;i--)
                cnm=cnm*i;
            for(int i=2;i<m+1;i++)
                cnm=cnm/i;
            return cnm;
        }
    private:
        int n;
        int r;
        int *a;
        int *b;
};
	\end{lstlisting}
	\begin{thebibliography}{99}
		\bibitem{book} 拓扑学,美 James R.Munkres著,熊金城,吕杰,谭枫译,机械工业出版社,2006年
		\bibitem{article} 关于有限集的拓扑种类问题,裴惠生,河南大学学报,1987年第四期
	\end{thebibliography}
\end{CJK*}
\end{document}
