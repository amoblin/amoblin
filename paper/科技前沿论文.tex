\documentclass[a4paper,12pt]{article}
\usepackage{CJKutf8}
\title{神经网络研究现状浅谈}
\author{崔贵林}
\date{\today}
\begin{document}
\begin{CJK*}{UTF8}{gbsn}
	\maketitle
	\begin{abstract}
		神经网络作为一门综合了脑科学,神经科学,认知科学,心理学,计算机科学,数学和物理学的交叉学科,在不断地发展着。神经网络研究在理论上向更加复杂的神经网络系统发展,在应用上范围不断扩展,在越来越多的领域得到应用。对神经网络的理论研究和实际应用进行材料的搜集,整理,归纳总结。
%	中文分词技术是中文信息处理的关键技术。目前,采用神经网络来进行
%		本文对神经网络在中文分词领域的研究进行材料的搜集,整理,归纳总结。整体上,神经网络
	\end{abstract}
	\emph{Keywords}:神经网络
	\newpage
	\section{引言}
	神经网络始于20世纪40年代,是人类智能研究的重要组成部分。它模拟人脑神经网络的结构和某些工作机制建立一种计算模型。
	\section{人工神经网络基本特征}
	(1)非线性。非线性关系是自然界的普遍特性。大脑的智慧就是一种非线性现象。人工神经元处于激活或抑制二种不同的状态,这种行为在数学上表现为一种非线性关系。具有阈值的神经元构成的网络具有更好的性能,可以提高容错性和存储容量。\\
(2)非局限性。一个神经网络通常由多个神经元广泛连接而成。一个系统的整体行为不仅取决于单个神经元的特征,而且可能主要由单元之间的相互作用、相互连接所决定。通过单元之间的大量连接模拟大脑的非局限性。联想记忆是非局限性的典型例子。\\
(3)非常定性。人工神经网络具有自适应、自组织、自学习能力。神经网络不但处理的信息可以有各种变化,
而且在处理信息的同时,非线性动力系统本身也在不断变化。经常采用迭代过程描写动力系统的演化过程。\\
(4)非凸性。一个系统的演化方向,在一定条件下将取决于某个特定的状态函数。例如能量函数,它的极值相应于系统比较稳定的状态。非凸性是指这种函数有多个极值,故系统具有多个较稳定的平衡态,这将导致系统演化
的多样性。
	\section{人工神经网络模型}
	人工神经网络模型主要考虑网络连接的拓扑结构、神经元的特征、学习规则等。\\
	目前已有近 40 种神经网络模型,其中有反传网络、感知器、自组织映射、Hopfield 网络、波耳兹曼机、适应谐振理论等。根据连接的拓扑结构,神经网络模型可以分为:\\
	(1)前向网络。 网络中各个神经元接受前一级的输入,并输出到下一级,网络中没有反馈,可以用一个有向无环路图表示。这种网络实现信号从输入空间到输出空间的变换,它的信息处理能力来自于简单非线性函数的多次复合。\\
	(2)反馈网络。网络内神经元间有反馈,可以用一个无向的完备图表示。这种神经网络的信息处理是状态的变换,可以用动力学系统理论处理。系统的稳定性与联想记忆功能有密切关系。Hopfield 网络、波耳兹曼机均属于这种类型。

%\section{理论研究}
%第二届神经网络国际会议(The Second International Symposium on Neural Networks, ISNN2005 )于5月30日至6月2日在重庆大学召开。
%第3届神经网络国际会议(International Symposium on Neural Networks 2006,简称ISNN2006)于2006年5月28日至6月1日在成都皇冠假日酒店召开。会议由电子科技大学与香港中文大学共同主办。香港中文大学王钧教授担任大会总主席。大会围绕着神经智能中的前沿理论和实际操作问题展开探讨,从3个专题:“神经网络理论”、“神经网络模型”和“算法和神经网络应用”,21个方面:”神经网络的数学理论“、”近似理论“、”学习理论“、”适应性共振理论“、”稳定性理论“、”径向基函数网络“,”递归神经网络“,”自组织映射“,”决策向量基”,“学习算法“,”信源盲分离“,”主成分分析和独立成分分析“,”信号与图象处理“、”机器人技术和控制“、”电信系统“,”时序分析“,”模式识别“,”金融工程”,“计算机安全”,“生物信息学”,“生物医学“,组织与会代表对神经网络领域的研究成果进行了深入的探讨。
%第4届神经网络国际研讨会(The 4th International Symposium on Neural Networks,简称ISNN2007)于2007年6月3~7日在南京召开。
%2005年

	\section{应用举例}
	\subsection{数据加密}
	在计算机网络大量普及的今天,信息本身就是时间,就是财富。信息传输,目前通过脆弱的信息通道,信息存储在“不设防”的计算机系统中,如何保护信息安全,使之不被窃取及不被篡改或破坏,已成为当今信息产业界普遍关注的重大问题。采用密码是有效而可行的办法。利用密码技术是“保护自己”的最简单办法。\\
加密的目的就是把原来看来很明白的“文本”加工成无法辨认的“乱码”。我们来研究一种工具。它把明文做成乱码。从历史来看, 公元前 4 世纪就有人探索过, 应用了。1977年有DES算法(数据加密标准)。1984年,有RSA加密算法。1999年,第一次提出用“人工神经网络”方法,解决加密新算法。\\
其实,数据加密和人工神经网络都可以看作一种变换。
	\subsection{交通信号灯}
	红绿灯设在十字路口或在多干道的岔口上,是为了调整岔口的交通秩序,由于不同时刻的车辆流通状况是复杂多变、高度非线性、随机的,还经常受人为因素的影响,非常难于获取精确的十字路口交通动态数学模型。交警的判断决策过程也难用简单的程序实现,用传统的常规闭环控制红绿灯达到最佳状态是非常困难的,因为传统的诸多控制方法都是建立在精确数学模型的基础上来实现的;而模糊控制正是建立在模糊概念上模仿人脑决策的控制理论,其鲁棒性强,尤其适用于非线性、时变、滞后的控制。所以采用模糊神经网络控制可以把模糊控制和神经网络两种技术的优点结合起来,既可以利用专家的经验知识,又具有学习逐步优化功能,特别适用于实时多变的交通状况,其控制效果优于一般智能控制方案。
	\subsection{车道保持控制系统}
	车道保持控制系统可在高度危险状态下及时做出响应,短时自主控制有可能偏离车道的汽车的行驶方向,使汽车沿着原先行驶的车道运行,从而避免交通事故的发生。首先建立仿真系统模型,包括方向盘,转向柱,转向中间轴,转向输入轴,齿轮齿条转向器和转向横拉杆等。施加在方向盘上的角驱动,是整车模型中的输入变量,通过该输入变量控制方向盘的转角,转角值通过车道保持系统模块来确定。汽车行驶时 , 机械系统动力学模型实时地将车速、偏航角和侧偏距离等信息输入到控制模块,控制模块根据采集到的信息和控制策略判断汽车的运行状态,决策出最佳方向盘转角,输出给机械模型,控制方向盘使汽车保持稳定行驶状态。\\
	为实现车道保持 , 需根据前面建立的输入输出变量设计控制器的算法 。可以通过PID控制方向盘转角。由于被控系统具有复杂的非线性特征且工作过程多具有时变不确定性,要想取得好的控制效果,就必须通过调整比例、积分和微分等PID的3个控制参数 ,形成控制变量中既相互配合又相互制约的关系,可以从复杂的非线性组合中找出最佳制约关系。研究表明,常规PID控制器由于不能实现控制参数的在线实时自整定调整 ,因此控制效果不佳 , 而利用BP神经网络具有的对环境变化的学习和适应能力及对强非线性控制等优点,则可以很好地实现对上述3个控制参数的在线调整,从而建立符合汽车强非线性、时变不确定性及滞后性等特点的自调整参数的PID控制系统。
	\subsection{中文分词}
	在英文的词句中,单词之间是以空格作为自然分界符的 ,而中文只是字、句和段可以通过明显的分界符来简单划界 , 唯独词没有一个形式上的分界符,词与词之间无明显分隔标记, 词的定义、词与词组划界标准与形式语法的缺乏等特点,构成了汉语词自动切分的极大困难 。\\
之前国内公开的分词系统采用的分词方法主要是:\\
1)机械切分: 运用简单的模式匹配技术的无条件切分。\\
2)语义切分: 对语句中的词义进行分析, 对自然语言自身的语言信息进行更多的处理。\\
%3 )人工智能切分: 模拟人的思维, 采用各种语义知识进行有条件的切分。\\
衡量一个自动分词系统的指标主要有三个:切分速度 、切分精度、系统的可维护性。而现在的分词方法,普遍存在切分效率不高 、不能有效消除歧义等缺陷。\\
尹峰等在1996年提出了以神经网络理论( BP 模型)为基础的汉语分词模型,为汉语分词研究开辟了新途径。以 非线性并行处理为主流的神经网络理论的发展为汉语自动分词研究开辟了新途径。\\
基于神经网络的知识表示方法不需要组织大量的产生式规则,以自组织,自学习的方式进行处理。这是一种 非逻辑非线性的全新的智能信息处理方法。神经网络分词系统还具有学习功能,它可根据用户的要求随意地增添或删除某些权重链接以达到维扩知识库的目的。\\
在实用中,BP算法存在收敛速度慢、易陷入局部最小等缺点,严重妨碍了分词速度。可以使用Levenbery—Marquart算法来加速收敛速度。\\
	应用神经网络来进行中文自动分词以实现智能化的中文自动分词系统,将分词知识的隐式方法存入神经网内部,通过自学习和训练修改内部权值,以达到正确的分词结果。\\
	神经网络分词法的关键在于知识库权重链表的组织和网络推理机制的建立。建立的分词过程是一个生成分词动态网的过程,该过程是分步进行的:首先以确定的待处理语句的汉字串为基础,来确定网络处理单元;然后,根据链接权重表激活输入/输出单元之间的链接,该过程可以采用某种激活方式,取一个汉字作为关键字,确定其链接表,不断匹配。\\
	2010年7月郑州航空工业管理学院杨文涛在基于BP神经网络的基础上使用改进的遗传算法。遗传算法 ( Generation Algorithm ) 是一种模拟生物界自然选择和自然遗传机制的高度并行、随机、自适应优化搜索算法。它具有隐含的并行性和对全局信息的有效搜索能力,采用概率化的寻优方法, 能自动获取和指导优化的搜索空间,自适应地调整搜索方向,不需要确定的规则,遗传算法尤其适合于处理传统搜索方法解决不了的复杂非线性问题,并发展成为现代有关智能计算中的关键技术。\\
	GA - BP 算法就是在BP算法之前,先用GA在某一点集中遗传出优化初值, 以此作为 BP算法的初始权值,再由BP算法进行训练,而后运用到BP神经网络控制,这就是GA-BP算法的基本原理。形成这种混合的GA-BP算法,解决BP网络容易陷入局部极小的问题同时提高分词过程中的收敛速度,同时可以发挥神经网络的概括映射能力,从而达到优化网络的目的。 
	\section{结束语}
	整体上,我国神经网络的研究处于缓慢而稳定的发展阶段,神经网络的应用范围不断扩展,在模式识别,信号处理,自动控制,人工智能,自适应人机接口和通信等方面都有应用。在基础科学领域,如高能物理的粒子跟踪,物理化学中的质谱分析,数学中的数值逼近及模式分类等;在共业生产领域,如机器人的控制,设备故障诊断,生产质量控制,产品分类和检验等;在医学工程领域,如脑电图,心电图的分类,若干疾病的早期诊断等;在国际军事领域,如飞行目标的识别与跟踪,武器系统的自动控制,导弹的实时制导等,在经济领域,如投资决策的分析与评估,证券股票涨落趋势的预测,石油价格的预测,银行支票签名的验证等。我们应该深入学习神经网络的理论知识,发掘其更广泛的应用,使神经网络更好地造福人类。
\end{CJK*}
\end{document}
