\documentclass[a4paper,12pt]{article}
\usepackage{amsmath}
\usepackage{CJKutf8}
\begin{document}
\begin{CJK*}{UTF8}{gbsn}
	论结果。\\
	揭示了方法存在着不收敛的固有缺陷!\\
	Jia&Stewart01更细致地讨论了方法的收敛性问题,对Jia95的结果进行了推广。\\
	结果适用于Arnoldi方法\\
	此时,可以证明(Saad80,Saad82,Jia95),通常A的“外部”(实部最大或最小,模最大,等)特征值一般先收敛。\\
	和Lanczos方法一样,我们有类似的判断收敛的准则。\\
	定理4.6.2\\
	\begin{equation}
		||r_i^{(m)}|| = ||(A-\mu_i^{(m)}I)x_i^{(m)}|| = h_{m+1m}|e_m^Ty_i^{(m)}|.
	\end{equation}
	说明:\\
	1.方法收敛时不一定要求$h_{m+1m}$小,只要$y_i^{(m)}$的最后一个分量很小,则$||r_i^{(m)}||$<++>很小。\\
	2.收敛前不必显式形成$x_i^{(m)}=V_my_i^{(m)}.$<++>
	基本Arnoldi算法:\\
	1.给定单位长度向量初始向量$v_1$<++>,要计算按某种顺序排列的k个特征对$(\lambda_i,x_i),i=1,2,\ldots,k,\varepsilon$<++>是预先给定的精度,赋值$m\leftarrow k.$<++>\\
	2.执行m步Arnoldi过程,得到$H_m,V_m.$\\
	3.计算$H_m$的特征对$(\mu_i^{(m)},y_i^{(m)})$,对应排序,用$\mu_i^{(m)}$逼进$\lambda_i,i=1,2,\ldots,k.$\\
	4.如果所有的$||r_i^{(m)}|| = h_{m+1m}|e_m^Ty_i^{(m)}|\le \varepsilon$,则形成$x_i^{(m)} = V_my_i^{(m)},i=1,2,\ldots,k,$停机;否则,$m\leftarrow m+1$,返回2.\\
	m大时,算法可能仍不收敛,若继续增大,则引起存储量的不可承受,计算量也急剧增大,和$m^2$成正比。\\
	怎么办?\\
	如果m大时仍不收敛,使用重新启动技术,如何从已有的信息构造新的初始向量$v_1$重新启动?这比方程组的算法远为复杂。eigs.m!\\
	在实际应用中,大量问题的小特征值很密集,或者要计算内部特征值,此时方法遇到本质性困难!\\
	4.7 精化的Rayleigh-Ritz方法,精化的Lanczos方法和精化的Arnoldi方法\\
	前面的结果表明,当子空间足够好时,Ritz值肯定收敛,但Ritz向量却可能收敛很慢,甚至不收敛,从而造成基本算法和重新启动算法的可能失败或效率很差。怎么办?\\
	精化的Rayleigh-Ritz方法(Jia94,97):\\
	给定子空间$\kappa$,假设$\mu$是$\lambda$的近似值,寻找满足下面最优性条件的单位长度向量$\hat{x} \in K:$\\
	\begin{equation}
		||(A-\mu I)\hat{x}|| = min_{\mu \in \kappa,||\mu||=1}||(A-\mu I)\mu||,
	\end{equation}
	用它逼近特征向量x,称之为精化的特征向量近似。\\
	$\mu$可以是任意合理的近似特征值,如Ritz值,调和Ritz值,调和Ritz向量的Rayleigh商!\\
	和经典的Rayleigh-Ritz方法及调和Rayleigh-Ritz方法一道,精化的Rayleigh-Ritz方法自2000年起被国际上三本权威著作列为解大规模矩阵特征问题的三类方法之一,得到广泛研究,被推广到求解很多其他问题,如计算大规模奇异值分解,广义奇异值分解,多项式特征值问题,伪特征值,加速大规模线性方程组的数值解法,google,等等。\\
	定理4.7.1 (Jia99,Jia&Stewart01,Jia04).如果$\mu \rightarrow \lambda$且$sin\angle(x,\kappa)\rightarrow 0$,则\\
	(1) 精化的Rayleigh-Ritz方法收敛
	(2) $\hat{x}$能唯一确定
	(3) 令$\mu,\hat{x}$为Ritz对,只要$||(A-\mu I)\tilde{x}|| \ne 0$,则必有\\
	\begin{equation}
		||(A-\mu I)\hat{x}|| \l ||(A-\mu I)\tilde{x}||;
	\end{equation}
	当有其它的近似特征值和$\mu$接近时,
	\begin{equation}
		||(A-\mu I)\hat{x}|| \ll ||(A-\mu I)\tilde{x}||,
	\end{equation}
	即精化方法比对应的经典方法远为精确。\\
	当取$\kappa$为Krylov子空间时,得到精化的Lanczos方法和精化的Arnoldi方法。以精化的Arnoldi方法为例,回忆m步Arnoldi过程
	\begin{equation}
		AV_m = V_mH_m + h_{m+1m}v_{m+1}e_m^T = V_{m+1}\hat{H}_m$,
	\end{equation}
	可以证明\\
	定理4.7.2(Jia94,97).
	\begin{equation}
		||A-\mu_i^{(m)}I)\hat{x}_i^{(m)}|| = \sigma_{min}(\hat{H}_m - \mu_i^{(m)}\tilde{I}),
		\hat{x}_i^{(m)} = V_mz_i^{(m)},
	\end{equation}
	其中$\tilde{I}$是m阶单位矩阵再加一行零元素,$z_i^{(m)}$是$\tilde{H}_m - \mu \tilde{I}$对应于最小奇异值的右奇异向量。\\
	注1:精化的特征向量近似能可靠地计算。\\
	注2:上式可以作为廉价可靠的判断收敛准则,收敛前不用显式形成$\hat{x}_i^{(m)} = V_mz_i^{(m)}$.\\
	注3: m步Arnoldi过程需要m个矩阵A乘向量和$2nm^2flops$,因此精化方法计算上可行,代价低廉,增加的$O(m^3)$flops可以忽略不计。\\
	证明:令$C^m$为m维空间,我们有
	\begin{equation}
		||(A-\mu_i^{(m)}I_)\hat{x}_i^{(m)}|| = min_{\mu\in\kappa,||\mu||=1}||(A-\mu_i^{(m)}I)\mu||
		=min_{z\in C^m,||z||=1}||(A-\mu_i^{(m)}I)V_mz||
		=min_{z \in C^m,||z||=1}||V_{m+1}(\tilde{H}_m - \mu_i^{(m)}\tilde{I})z||
		= min_{z \in C^m,||z||=1}||(\tilde{H}_m - \mu_i^{(m)}\tilde{I})z||
		= ||(]tilde{H}_m - \mu_i^{(m)}\tilde{I})z_i^{(m)}||
		= \sigma_{min}(\tilde{H}_m - \mu_i^{(m)}\tilde{I}).
	\end{equation}
	基本的精化Arnoldi算法:\\
	1.给定单位长度向量初始向量$v_1$,要让计算按某种顺序排列的k个特征对$(\lambda_i,x_i),i=1,2,\ldots,k,\varepsilon$是预先给定的精度,赋值$m\leftarrow k$.\\
	2.执行m步Arnoldi过程,得到$H_m,V_{m+1}$.
	3.计算$H_m$的特征值$\mu_i^{(m)}$,对应排序,用$\mu_i^{(m)}$逼近$\lambda_i, i=1,2,\ldots,k;$计算$\tilde{H}_m - \mu_i^{(m)}\tilde{I}$对应于最小奇异值的右奇异向量$z_i^{(m)},i=1,2,\ldots,k.$\\
	4.如果所有的$\sigma_{min}(\tilde{H}_m - \mu_i^{(m)}\tilde{I}) \le \varepsilon$,则形成$\hat{x}_i^{(m)}=V_mz_i^{(m)},i=1,2,\ldtos,k$,停机;否则,$m\leftarrow m+1$,返回2.\\
	和Arnoldi算法类似,为了开发实用的算法,也要重新启动精化算法!\\
	广泛的实际计算表明,重新启动的精化Arnoldi算法(reigs.m)比对应的标准Arnoldi算法(eigs.m)可以快数十倍甚至更多!\\
	4.8 调和Rayleigh-Ritz方法,调和Lanczos方法和Arnoldi方法\\
	前面提到,在实际应用中,大量问题的小特征值很密集,或者要计算内部特征值,此时Lanczos方法和Arnoldi方法将对于计算小特征值收敛很慢,对于内部特征值将遇到本质和严重的困难。\\
	原因:1.方法对内部特征值效果很差,收敛很慢;2.方法无法可靠地挑选正确的Ritz值逼近所需要的特征值。\\
	内部特征值:点$\tao$附近的若干特征值和/或对应的特征向量。\\
	假设$\tao$不等于A的任意特征值,则$(A-\tao I)^{-1}$存在。由$Ax=\lambda x$可得
	\begin{equation}
		(A-\tao I)^{-1}x = \frac{1}{\lambda - \tao}x.
	\end{equation}
	因此,位移求逆矩阵$A-\tao I)^{-1}$将A的内部特征值转换成$(A-\tao I)^{-1}$的绝对值最大的特征值,并相隔较开。
	可以对$(A-\tao I)^{-1}$使用Lanczos方法和Arnoldi方法。\\
	然而,在用Arnoldi过程构造$\kappa(v_1,(A-\tao I)^{-1},m)$的标准基向量$V_m$时,每步都要计算$(A-\tao I)^{-1}v_i,i=1,2,\ldots,m,$这等价于解m个大规模的线性方程组。\\
	无论是用直接法还是迭代法,一般来说都太昂贵,不实际!\\
	企图:避免解方程组!\\
	如何实现?\\
	调和Rayleigh-Ritz方法(1991):\\
	给定m维的子空间$\kappa$,用满足投影条件\\
	$\{ x^{(m)} \in \kappa
	(A-\mu^{(m)}I)x^{(m)} \bot (A-\tao I)\kappa $
	的$(\mu^{(m)},x^{(m)})$作为A的部分特征值和特征向量的近似,称之为A在$\kappa$上关于$\tao$的调和Ritz值和调和Ritz向量。\\
	取$\kappa$为Krylov子空间,则对对称和非对称问题分别得到调和Lanczos方法和调和Arnoldi方法。\\
	以调和Arnoldi方法为例,回忆Arnoldi过程,可证上述投影等价于m阶广义特征问题
	$\{(\tilde{H}_m - \tao \tilde{I})^H(\tilde{H}_m - \tao\tilde{I})y^{(m)} = (\mu^{(m)} - \tao)(H_m - \tao I)^Hy^{(m)}
	x(m) = V_my^{(m)}$
	如果要求最靠近$\tao$的k个特征值,则用最靠近$\tao$的k个调和Ritz值逼近。\\
	类似地,可以写出基本的调和Arnoldi算法。\\
	直观上,目标点$\tao$越接近某个特征值,则方法应该收敛越快,但实验中的性态并非如此,很复杂。\\
	收敛性分析比Rayleigh-Ritz方法更困难更复杂,虽然国际上很多学者一直在研究,但结果很零散,特殊,很多令人费解和迷惑的现象无法解释,直到最近才有了具有普遍意义的一般性结果。\\
	定理4.8.1(Jia05). 假设$sin\angle(x,\kappa)\rightarrow 0$,则\\
	1.如果$\tao$离$\lambda$远,则方法收敛慢;如果离的太近,则方法将丢失$\lambda$,即调和Ritz值$\mu^{(m)}$不会逼近$\lambda$.\\
	2.调和Ritz向量的Rayleigh商$\row = (x^{(m)})^HAx^{(m)}$是比$\mu^{(m)}$更可靠更精确的近似。\\
	3.调和Ritz向量$x^{(m)}$收敛可能很慢,甚至不收敛。\\
	4.即使$sin\angle(x,\kappa)=0$,即$x\in \kappa$,方法也可能完全失败。\\
	如何克服方法的可能不收敛缺陷?\\
	使用精化Rayleigh-Ritz方法的原理:\\
	取精化方法中的$\mu$为上述$\row$,在子空间中寻找相应的精化特征向量近似。开发的算法效果显著,高效可靠,比原来的算法能快数十倍乃至更多!\\
	5 多项式差值和数值逼近\\
	在科学和工程实验中会给出区间[a,b]上n+1个点$a\le x_0 \l x_1 \l \ldots \l x_n \le b$和函数值$f(x_i)$。数组$(x_i,f(x_i)),i=0,1,\ldots,n$有具体的物理意义,如在时间或温度$x_i$时对应速度,人口或压力$f(x_i)$;函数本身可能未知。\\
	由这些数据,我们可能需要\\
	(1) 画出一条光滑曲线,通过者写数据点,以推测区间内点之间的实验结果\\
	(2) 用某个容易简单的函数整体上捕获f(x)的性态,在某种意义下使得它拟合复杂函数f(x)。\\
	(3) 逼近f(x)的导数或积分,尽快地高精度地算出它们。\\
	5.1 多项式插值和病态性\\
	简言之,插值就是用某个简单函数拟合已知或未知的f(x),要求在这些点处,插值函数和被插值f(x)在n+1个点$x_i$处的值相等。
	定义5.1.1 给定n+1个数据$(x_i,f(x_i)),i=0,1,\ldots,n,$其中$x_i \in [a,b],f(x)$是[a,b]上的实值函数,寻找一个具有n+1个参量的函数$\phy(x;c_0,c_1,\ldots,c_n)$使得$\phy(x_i;c_0,c_1,\ldots,c_n) = f(x_i),i=0,1,\ldots,n.$\\
	简记$\phy(x)=\phy(x;c_0,c_1,\ldots,c_n),$称之为f(x)在[a,b]上的插值函数,如果$\phy(x)\in \Uppsi=span\{\phy_0(x),\phy_1(x),\ldots,\pyh_n(x)\}$关与$c_0,\ldots,c_n$线性,则$\phy(x) = c_0\phy_0(x) + c_1\phy_1(x) + \ldots + c_n\phy_n(x)$.\\
	如果基函数$\phy_k(x)$为代数多项式,则$\phy(x)$为多项式插值;如果$\phy_k(x)$为有理函数,则$\phy(x)$为有理函数;如果$\phy_k(x)$为三角函数,则$pyh(x)$为三角函数插值。\\
	我们着重讲多项式插值,讨论解的存在行,唯一性和稳定性。\\
	插值函数
	\begin{equation}
		\phy(x) = \sum\limits_{k=0}^{n}c_k\phy_k(x)
	\end{equation}
	中的系数$c_k$待定,由插值条件可得
	\begin{equation}
		\phy(x_i) = \sum\limits_{k=0}^{n}c_k\phy_k(x_i) = f(x_i), i=0,1,\ldots,n
	\end{equation}
	这可写成线性方程组Ac = y,其中
	$A=(
	\begin{array}{cccc}
		\phy_0(x_0) & \phy_1(x_0) & \ldots &\phy_n(x_0) \\
		\phy_0(x_1) & \phy_1(x_1) & \ldots &\phy_n(x_1) \\
		\ldots	&	\ldots	&	\ldots	&	\ldots	\\
		\phy_0(x_n) & \phy_1(x_n) & \ldots &\phy_n(x_n) \\
	\end{array},
	c=(c_0,c_1,\ldots,c_n)^T$和$y=(f(x_0),f(x_1),\ldots,f(x_n))^T$.\\
	在有限精度下,计算出来c的精度依赖于A的性态。如果A的条件数很大,则解的精度可能很差,导致插值函数$\phy(x)$精确度很差,所以应该选择基函数使得导出的A不病态。\\
	常用的选择:\\
	(1) 单项式基$\phy_j(x)=x^j,j=0,1,\ldots,n.$\\
	插值多项式$p_n(x)$可写成$p_n(x)=c_0+c_1x+\ldots+c_nx^n$.\\
	A是一个Vandemonde矩阵,当有两个数据点$x_i$靠近时或n大时,A高度病态。\\
	(2) Lagrange插值\\
	要求基函数$l_j(x)$为n次多项式,且满足性质
	\begin{equation}
		l_j(x_k)=\{1,k=j\\0,k\neq j
	\end{equation}
	因此
	\begin{equation}
		l_j(x) = \prod\limits_{k=0,k\neq j}(x-xk)/\prod\limits_{k=0,k\neq j}(x_j-x_k),j=0,1,\ldots,n.
	\end{equation}
	此时A=I为单位阵,精度高,插值多项式
	\begin{equation}
		L_n(x)= \sum\limits_{j=0}^nl_j(x)f(x_j).
	\end{equation}
	记$\omega_{n+1)} = (x-x_0)(x-x_1)\cdots(x-x_n)$,\\
	则$L_n(x) = \sum\limits_{j=1}^n\frac{\omega_{n+1}(x)}{(x-x_j)\omega_{n+1}^'(x_j)}f(x_j)$.\\
	缺点:当增加一个插值点时,所有的基函数都要重新计算,没有有效的更新公式。
	(3) Newton插值\\
	对$j=0,1,\ldots,n,$取$\phy_0(x)=1,\phy_j(x)=(x-x_0)(x-x_1)\cdots(x-x_{j-1}),$\\
	则插值多项式可写成$p_n(x)=c_0+c_1(x-x_0)+\cdots+c_n(x-x_0)(x-x_1)\cdots(x-x_{n-1}).$\\
	根据插值条件$p_n(x_i)=f(x_i),i=0,1,\ldots,n,$可得关于系数$c_0,c_1,\ldots,c_n$的下三角方程组
	\begin{equation}
		p_n(x_0) = c_0 = f(x_0)
		p_n(x_1) = c_0 + c_1(x_1 - x_0) = f(x_1)
		\cdots	\cdots
		p_n(x_n) = c_0 + c_1(x_n-x_0)+\ldots+c_n(x_n-x_0)(x_n-x_1)\cdots(x_n-x_{n-1})=f(x_n).
	\end{equation}
	由此可逐个得到$c_0,c_1,\ldots,c_n$.用差商记号可表示为\\
	$c_0=f(x_0),c_1=\frac{f(x_1)-f(x_0)}{x_1-x_0}=f[x_0,x_1]$,一阶差商\\
	$c_2=\frac{1}{x_2-x_1}(\frac{f(x_2)-f(x_0)}{x_2-x_0}-c_1)$\\
	$=\frac{f[x_0,x_2]-f[x_0,x_1]}{x_2-x_1}=f[x_0,x_1,x_2]$,二阶差商\\
	一般地\\
	$c_j=f[x_0,x_1,\ldots,x_j]=\frac{f[x_0,x_1,\ldots,x_{j-2},x_j]-f[x_0,x_1,\ldots,x_{j-2},x_{j-1}]}{x_j-x_{j-1}},j=1,2,\ldots,n.$\\
	因此$p_n(x) = f(x_0)+f[x_0,x_1](x-x_0)+\ldots+f[x_0,x_1,\ldots,x_n](x-x_0)\cdots(x-x_{n-1}).$\\
	优点:每增加一个插值点,只增加一项,能廉价更新。\\
	截断误差:上述三种插值多项式虽然形式不同,但是结果是唯一的,因此,可以统一起来估计它们和原函数之间的误差:假设f(x)在[a,b]上n+1阶可导,则截断误差或插值余项
	\begin{equation}
		R_n(x)=f(x)-p_n(x)=\frac{f^{(n+1)}(\xi)}{(n+1)!}\omega_{n+1}(x),\xi \in [a,b]
	\end{equation}
	收敛性:有条件,否则不收敛!
	\begin{equation}
		|R_n(x)|=|f(x)-p_n(x)|=|\frac{f^{(n+1)}(\xi)}{(n+1)!}||\omega_{n+1}(x)|.
	\end{equation}
	条件一:$max_{x\in[a,b]}|f^{(n+1)}(x)|\le M_1$,有界或者不能随n增长太快\\
	条件二:$max\limits_{x\in[a,b]}|\omega_{n+1}(x)|\le M_2$,有界或者不能随n增长太快!\\
	例:
	\begin{equation}
		f(x) = \frac{1}{1+25x^2},x\in[-1,1],x_i=-1+\frac{2i}{n},i=0,1,\ldots,n.
	\end{equation}
	会发现在[-1,1]中间Lagranage插值多项式$p_10(x)$比$p_5(x)$能很好地逼近f(x),但在接近端点处,两者都发散,前者比后者偏离f(x).\\
	实际上,可证明,当、想|>0.726时,$p_n(x) \notrightarrow f(x)$.\\
	这是著名的Runge现象!\\
	策略:合理选择插值点$x_i,i=0,1,\ldots,n$使
	\begin{equation}
		max\limits_{x\in[a,b]}|\omega_{n+1}(x)|=min
	\end{equation}
	可能得到满意的结果!\\
	经典结果:在首项系数为1的n次多项式集合$bar{P}_n$中\\
	$min\limits_{q_n(x)\in\bar{P}_n}max\limits_{x\in[-1,1]}|q_n(x)| = max\limits_{x\in[-1,1]}|\frac{1}{2^{n-1}}T_n(x)|=\frac{1}{2^{n-1}},$\\
	其中$T_n(x)$是[-1,1]上的n次Cehbyshev多项式$T_n(x)=cos(narccosx).$\\
	性质一:三项递推式:\\
	$T_0(x) = 1,T_1(x)=x,T_n(x)=2xT_{n-1}(x)-T_{n-2}(x).$\\
	性质二:$T_n(x)$在[-1,1]上有n个零点。\\
	$x_i=cos(\frac{2i+1}{n}\frac{\pi}{2}),i=0,1,\ldots,n-1.$\\
	因此,若在[-1,1]上进行n次多项式插值,可能的话,取插值点为n+1次$T_{n+1}(x)$的零点,这将使得截断误差尽可能小。\\
	对于一般区间[a,b],可进行变量变换将其转化为[-1,1];$x=\frac{a+b}{2}+\frac{b-a}{2}t,$从而将插值点$x_i\in[a,b]$变为$t_i\in[-1,1]$,因此可应用上述结果。\\
	思考:变换后结果的形式是什么?\\
	稳定性:当$f(x_i)=\tilde{f}(x_i)+\delta_i$时,插值函数$I_n(f,x)=\sum\limits_{j=0}^{n}l_j(x)f(x_j)$的误差是否可以用$max|\delta_i|$控制?\\
	答案是否定的,因为$\varepsilon_n = max\limits_{x\in[a,b]}|I_n(f,x)-I_n(\tilde{f},x)|\le max\limits_{x\in[a,b]}\sum\limits_{j=0}^{n}|l_j(x)|max\limits_{0\le i\le n}|\delta_i|$.\\
	而$l_j(x)$对j有正有负,因此对大的n可能有$max\limits_{x\in[a,b]}\sum\limits_{j=0}^n|l_j(x)|\gg 1$.\\
	结论:高次多项式插值的稳定性不能保证,不能使用!\\
	5.2 最佳平方逼近和最小二乘拟合
	目标:和插值一样,用有限维空间中的简单函数$\phy(x)\in \Psi = span\{\phy_0(x),\phy_1(x),\ldots,\phy_n(x)\}$逼近f(x)\in[a,b],但现不要求$\phy(x_i)=f(x_i),i=1,2,\ldots,m$,而是在连续情况下寻找$\phy^*\in \Psi$使得
	$||f-\phy^*||_2^2 = min\limits_{\phy\in\Psi}\int_a^b[f(x)-\phy(x)]^2\row(x)dx$,
	在离散情况下(f(x)本身可能未知),对于m>n,$||f-\phy^*||_2^2 = min\limits_{\phy\in\Psi}\sum\limits_{i=0}^{m}[f(x_i)-\phy(x_i)]^2\row(x_i)$.\\
	$\phy^*(x)$称为最佳平方逼近和最小二乘拟合。$\row(x)\ge0$不恒等于零称之为权函数。当$\Psi$为多项式集合时,称为多项式最佳平方逼近和多项式最小二乘拟合。和插值函数不同,虽然可能不通过$(x_i,f(x_i))$,但最佳逼近$\phy^*(x)$在整个区间上在范数意义下最好地拟合原函数的性态,使整体误差最小。\\
	如何求解?注意连续和离散内积的定义:\\
	\begin{equation}
		(f,g)=\int_a^bf(x)g(x)\row(x)dx,
		(f,g) = \sum\limits_{i=0}^{m}f(x_i)g(x_i)\row(x_i).
	\end{equation}
	定理5.2.1 如果$\phy_0,\ldots,\phy_n$线性无关,则最佳平方逼近和最小二乘拟合问题
	$||f-\phy^*||_2 = min\limits_{\phy\in\Psi}||f-\phy||_2$
	有唯一解$\phy^* = \sum\limits_{j=0}^nc_j^*\phy_j$,
	其中系数$c_j^*$称为正交系数或Fourier系数,满足线性方程组(法方程组)
	$\sum\limits_{j=0}^n(\phy_j,\phy_k)c_j^*=(f,\phy_k),k=0,1,\ldots,n,$\\
	特别地,如果$\phy_0,\ldots,\phy_n$正交,则
	$c_j^* = (f,\phy_j)/(\phy_j,\phy_j),j=0,1,\ldots,n.$\\
	证明:令$c_0,c_1,\ldots,c_n$是一组数,且至少有一个$c_j\neq c_j^*$,则
	$\sum\limits_{j=0}^nc_j\phy_j-f = \sum\limits_{j=0}^n(c_j-c_j^*)\phy_j + \phy^* -f.$\\
	由法方程组可知$(\phy^*-f,\phy_k)=(\sum\limts_{j=0}^nc_j^*\phy_j-f,\phy_k) = 0,k = 0,1,\ldots,n,$
	因此$\phy^*-f$和$\sum\limits_{j=0}^n(c_j-c_j^*)\phy_j$正交,从而
	$||\sum\limits_{j=0}^nc_j\phy_j-f||_2^2 = ||\sum\limits_{j=0}^n(c_j-c_j^*)\phy_j||_2^2 + ||\phy^*-f||_2^2 \ge ||\phy^*-f||_2^2.$\\
	等式成立$\Leftrightarrow$全部的$c_j=c_j^*$,因此法方程组的解是最佳逼近和最小二乘拟合问题的解。\\
	解的唯一性:须证明当$\phy_0,\phy_1,\ldots,\phy_n$线性无关时,法方程组的系数矩阵非奇异,根据内积的对称性(f,g)=(g,f),法方程组系数矩阵显然是对称。如果它奇异,则齐次方程组
	$\sum\limits_{j=0}^n(\phy_j,\phy_k)c_j=0,k=0,1,\ldots,n$
	有非零解,即至少有一个$c_k\neq0$,此时
	$||\sum\limits_{j=0}^nc_j\phy_j||_2^2 = (\sum\limits_{j=0}^nc_j\phy_j,\sum\limits_{k=0}^nc_k\phy_k)
	= \sum\limits_{k=0}^n(\sum\limits_{j=0}^n(\phy_j,\phy_k)c_j)c_k
	= \sum\limits_{k=0}^n0·c_k=0,$
	与$\phy_0,\phy_1,\ldots,\phy_n$的线性无关性矛盾!\\
	注:当$\Psi$为n次多项式集合时,可以用漂亮的三项递推式(等价于对称矩阵的Lanczos过程)逐次构造计算它的一组正交多项式基$\{\phy_j\}_{j=0}^n$.\\
	当$\{\phy_j\}_{j=0}^n$正交时,最佳平方逼近和最小二乘拟合逼近的误差
	$||f-\phy^*||_2^2 = (f-\phy^*,f-\phy^*)
	=||f||_2^2-\sum\limits_{j=0}^nc_j^*(f,\phy_j)
	=||f||_2^2 - \sum\limits_{j=0}^n(c_j^*)^2||\phy_j||_2^2
	=||f||_2^2 - \sum\limits_{j=0}^n(\frac{(f,\phy_j)}{||\phy_j||_2})^2\ge 0$.\\
	Bessel不等式:
	\begin{equation}
		\sum\limits_{j=0}^n(c_j^*)^2||\phy_j||_2^2\le ||f||_2^2.
	\end{equation}
	因此,级数$\sum\limits_{j=0}^\infty(c_j^*)^2||\phy_j||_2^2$收敛。\\
	实际情况:\\
	1.当[a,b]=[-1,1],$\row(x)=\frac{1}{\sqrt{1-x^2}}$时,正交多项式为Cehbyshev多项式$T_j(x).(T_0,T_0)=\pi;$当j>0时,$(T_j,T_j)=\frac{\pi}{2}$.\\
	2.当[a,b]=[-1,1],$\row(x)=1$时,正交多项式为Legendre多项式,其三项递推式
	\begin{equation}
		\phy_0(x)=1,\phy_1(x)=x,
		\phy_n=\frac{2n-1}{n}x\phy_{n-1}(x)-\frac{n-1}{n}\phy_{n-2}(x).
		(\phy_j,\phy_j)=\frac{2}{2j+1}.
	\end{equation}

	6 非线性方程组的数值解法
	科学和工程科学计算中的大多数问题都是非线性问题,前面的矩阵特征值问题就是一个著名的例子,因为n阶矩阵的特征值是单变量的n次多项式方程的根,但线性方程组是典型的线性问题,非线性问题的数值解法一般来讲都必须是迭代法,很少有直接法存在。\\
	非线性问题中的至少一个变量的次数大于一。\\
	可写成如下实值函数的求解问题F(x)=0,
	其中$x=(x_1,\ldots,x_n)^T\in D$
	映射$F=(f_1(x),\ldots,f_n(x))^T:D\belongs R^n \rightarrow R^n$.\\
	难度:可能比线性问题难度大得多。\\
	原因:存在性唯一性可能很复杂,和前面讲的问题不同,对于一般的非线性问题,可能无解,有唯一解,无穷多解。但一般没有解析表达式,不知道解集合。\\
	例子:\\
	1.$e^x + 1 =0$,无解
	2.$e^{-x} -x = 0$,唯一解
	3.$x^2-4sinx=0$,两个解
	4.$x^3+6x^2+11x-6=0$,三个解
	5.$sinx=0$,无穷多个解,$x_k=k\pi,k=0,1,\ldots,n.$
	$F(x) = (
	\begin{array}{c}x_1^2-x_2+\alpha\\-x_1+x_2^2+\alpha\end{array})=(\begin{array}{c}0\\0
	\end{array}$.
	当$\alpha=1,0.25,0,-1$时分别无解,一个解,两个解,四个解。\\
	鉴于解的存在性唯一性的复杂性,我们只能满足于在某特定区域$D\belongs R^n$上找到F(x)=0的解的存在性和唯一性条件,然后在此区域上寻找开发求解的可行性算法。\\
	由于问题的非线性性,只考虑迭代法。\\
	显然,对于初始向量$x_0\in D$,必须要求迭代序列${x_k}\in D$,此称为迭代序列的适定性,否则会破坏F(x)=0在D上解的存在性唯一性,产生不可知或没有意义的结果。\\
	同时,我们必须研究迭带序列$\{x_k\}$的收敛速度,以找到尽可能快的迭代法。\\
	6.1 向量值函数的导数及性质\\
	研究F(x)=0的求解必须涉及到F(x)的导数。\\
	类似于单变量函数的连续性和导数,我们有\\
	定义6.1.1 设$F:D\belongs R^n \rightarrow R^m$,若对于任意的$\varepsilon > 0$,存在
	$\delta >0,x\in S(x_0,\delata)=\{x:| ||x-x_0|| < \delta\}\belongs D,$
	使得$||F(x)-F(x_0)||<\varepsilon$,则称F(x)在$x_0\in int(D)$连续,若$F(x)$在D内任一点连续,则称F(x)在D内连续。进一步,对于任意的$\varepsilon>0$,存在$\delta>0$,使得对任意的$x,y\in D,||x-y||<\delta$,则有$||F(x)-F(y)||<\varepsilon$,则称F(x)在D上一致连续。\\
	设$F:D\belongs R^n\rightarrow R^m$是闭区域$D_0\belongs D$上的映射,若对任意的$x,y\in D_0$,存在常数L,使得$||F(x)-F(y)||\le||x-y||,$则称F(x)在$D_0$上Lipschitz连续.
	定义6.1.2 设$F:D\belongs R^n\rightarrow R^m$,对$x\in int(D)$和任意的$h \in R^n$,若存在$A(x)\in R^{m\times n}$,有$lim_{h\rightarrow 0}\frac{||F(x+h)-F(x)-A(x)h||}{||h||}=0,$则称F(x)在x处可导,A(x)称为F(x)在x处的导数,记为$F^'(x)=A(x)$.若F(x)对任意的$x\in int(D)$可导,则称F(x)在D内可导。
	注:上述两个定义中的凡属可任选,因为在有限维空间中范数等价,唯一的差别是L的大小会不同。\\
	$F^'(x)=A(x)$称为F(x)的Jacobi矩阵,可以验证
	\begin{equation}
		F^'(x) = (
		\begin{array}{cccc}
			\frac{\partial f_1(x)}{\partial x_1} & \frac{\partial f_1(x)}{\partial x_2} & \ldots & \frac{\partial f_1(x)}{\partial x_n} \\
			\frac{\partial f_2(x)}{\partial x_1} & \frac{\partial f_2(x)}{\partial x_2} & \ldots & \frac{\partial f_2(x)}{\partial x_n} \\
			\vdots & \vdots & \ldots & \vdots \\
			\frac{\partial f_m(x)}{\partial x_1} & \frac{\partial f_m(x)}{\partial x_2} & \ldots & \frac{\partial f_m(x)}{\partial x_n} \\
		\end{array}
		)
		=(
		\begin{array}{c}
			\bigtriangledown f_1(x)^T\\
			\bigtriangledown f_2(x)^T\\
			\vdots \\
			\bigtriangledown f_m(x)^T\\
		\end{array}
		)
		=\bigtriangledown F(x)^T.
	\end{equation}
	$\bigtriangledown F(x)$是F(x)的梯度。
	定理6.1.1 设$F:D\belongs \REAL^n \rightarrow \REAL^m$在开凸区域D上可导,则\\
	(1) F(x)在D上连续\\
	(2) 对任意的$x,y\in D$,有$||F(y)-F(x)||\leq sup\limits_{0\leq t\leq 1}||F^'(x+t(y-x))||||y-x||$.
	$||F(y)-F(x)-F^'(x)(y-x)||\leq sup\limits_{0\leq t\leq 1}||F^'(x+t(y-x))-F^'(x)||||y-x||.$\\
	(3) 特别地,如果$F^'(x)$在D上Lipschitz连续,则$||F(y)-F(x)-F^'(x)(y-x)||\leq \frac{L}{2}||y-x||^2$,其中L为Lipschitz常数.
	证明:(1) 对任意的$x\in D$,因为$lim_{h\rightarrow 0}\frac{F(x+h)-F(x)-F^'(x)h}{||h||}=0,$所以$lim_{h\rightarrow 0}F(x+h) = lim_{h\rightarrow 0}F(x) + F^'(x)h = F(x).$

\end{CJK*}
\end{document}
