\documentclass[a4paper,12pt]{article}
\usepackage{CJKutf8}
\usepackage{mathabx}
\title{泛函的泛想}
\author{崔贵林2009020268}
\date{\today}
\begin{document}
\begin{CJK*}{UTF8}{gbsn}
	\maketitle
	\section{引言}
	\quad\quad有了空间的概念以后,我们开始思考空间和空间之间的变换,这种变换我们称作算子。\\
	然后有两种延伸:1. 线性空间到线性空间的变换,如果满足线性性,那么就叫做线性算子。2.我们最为熟悉的当然是实数空间了,于是我们希望将空间和实数联系起来,这就是我们所研究的泛函。\\
	综合上述两种,满足线性性的泛函,就是线性泛函。\\
	我们也需要从空间的角度去研究算子和泛函,于是:\\
	Banach空间到完备的Banach空间的有界线性算子的全体组成的集合,可以构成一个线性空间,继而又赋予算子范数的概念,证明其完备性,得到其为Banach空间。\\
	具体到泛函,我们就得到了共轭空间。\\
	由于共轭空间是一个Banach空间,重复上述操作,得到共轭空间的共轭空间,就是原空间的第二共轭空间,研究原空间和第二共轭空间,我们发现了某些相似性。若原空间和第二共轭空间同构,称原空间为自反空间。\\
	Y的共轭空间到X的共轭空间的算子是X到Y的算子的共轭算子。\\
	研究共轭空间中的极限和原Banach空间中极限的关系,我们得到了弱收敛。\\
	研究第二共轭空间中的极限和共轭空间中的极限的关系,我们得到了*弱收敛。\\
	\section{次线性泛函}
	\quad\quad设$\mathcal{X}$是线性空间,C是$\mathcal{X}$上含有$\theta$的凸子集,在$\mathcal{X}$上规定一个取值于$[0,\infty]$的函数
	$$P(x) = inf\{lambda>0|\frac{x}{\lambda}\in C\} (\forall x \in \mathcal{X})$$
	与C对应,称函数P为C的Minkowski泛函。\\
	关于Minkowski泛函有下面的性质。\\
	设$\mathcal{X}$是线性空间,C是$\mathcal{X}$上含有$\theta$的凸子集。若P为C的Minkowski泛函,则P具有以下性质:
	\begin{enumerate}
		\item $P(x) \in [0,\infty], P(\theta) = 0;$
		\item $P(\lambda x) = \lambda P(x) (\forall x \in \mathcal{X},\forall \lambda > 0)$;
		\item $P(x+y) \le P(x) + P(y) (\forall x,y \in \mathcal{X})$.
	\end{enumerate}
	闵科夫斯基泛函是一种次线性泛函。我们要寻找线性泛函。
	\section{线性泛函}
	\quad\quad首先我们先看线性算子。\\
	设$\mathcal{X},\mathcal{Y}$是两个线性空间,D是$\mathcal{X}$的一个线性子空间,$T:D\to \mathcal{Y}$是一种映射,D称为T的定义域,有时记做D(T).$R(T) = \{ Tx | \forall x \in D\}$称为T的值域。如果
	$$T(\alpha x + \beta y) = \alpha Tx + \beta Ty (\forall x,y \in D,\forall \alpha,\beta \in K),$$
	那么称T是一个线性算子。\\
	而取值于实数(复数)的线性算子称为实(复)线性泛函,记做f(x)或$<f,x>$(即线性函数)。\\
	用$\mathcal{L}(\mathcal{X},\mathcal{Y})$表示一切由$\mathcal{X}$到$\mathcal{Y}$的有界线性算子的全体,并规定
	$$\|T\| = \sup_{x \in X - \theta}\|Tx\|/\|x\| = \sup_{\|x\|=1}\|Tx\|$$
	为$T \in L(x,y)$的范数,特别用$\mathcal{L}(\mathcal{X})$表示$\mathcal{L}(\mathcal{X},\mathcal{X})$及用$\mathcal{X}^*$表示$\mathcal{L}(\mathcal{X},K)$,即$\mathcal{X}^*$表示$\mathcal{X}$上的线性有界泛函全体。\\
	接下来的定理说明了$\mathcal{L}(\mathcal{X},\mathcal{Y})$的完备性。\\
	设$\mathcal{X}$是$B^*$空间,y是B空间,若在$\mathcal{L}(\mathcal{X},\mathcal{Y})$上规定线性运算:
	$$(\alpha_1 T_1 + \alpha_2 T_2)(x) = \alpha_1 T_1 x + \alpha_2 T_2 x (\forall x \in \mathcal{X}),$$
	其中$\alpha_1,\alpha_2 \in K, T_1,T_2\in \mathcal{L}(\mathcal{X},\mathcal{Y}),$则$\mathcal{L}(\mathcal{X},\mathcal{Y})$按$\|T\|$构成一个Banach空间。\\
	\section{共轭空间,第二共轭空间与自反空间}
	\quad\quad设$\mathcal{X}$是一个$B^*$空间,$\mathcal{X}$上的所有连续线性泛函全体,按范数
	$$\|f\| = \sup_{\|x\|=1}\|f(x)\|$$
	构成一个B空间,称为$\mathcal{X}$的共轭空间。
	$\mathcal{X}^*$的共轭空间,记做$\mathcal{X}^{**}$,称为$\mathcal{X}$的第二共轭空间。$\forall x \in \mathcal{X}$,可以定义
	$$X(f) = <f,x>a (\forall f \in \mathcal{X}^*).$$
	其实这里就有哲学的辩证思想。$X(f)=f(x)$。当f不变,x在变化时,就是我们最常见的f(x);而当x不变,而f在变化时,就是X(f)。为什么f能变化呢?因为我们取的是$X^*$空间。
	不难验证:X还是$\mathcal{X}^*$上的一个线性泛函,满足:
	$$|X(f)| \le \|f\|\|x\|.$$
	从而X还是连续的,满足
	$$\|X\| \le \|x\|.$$
	定理:$B^*$空间$\mathcal{X}$与它的第二共轭空间$\mathcal{X}^{**}$的一个子空间等距同构。\\
	定义:如果$\mathcal{X}$到$\mathcal{X}^{**}$的自然映射T是满射的,则称$\mathcal{X}$是自反的,记做$\mathcal{X}=\mathcal{X}^{**}.$
	\section{共轭算子}
	\quad\quad设$\mathcal{X},\mathcal{Y}$是$B^*$空间,算子$T\in \mathcal{L}(\mathcal{X},\mathcal{Y}).$算子$T*:\mathcal{Y}^*\to\mathcal{X}^*$称为是T的共轭算子是指:
	$$f(Tx) = (T^*f)(x) (\forall f \in Y^*,\forall x\in X).$$
	共轭算子,顾名思义,是在共轭空间上的算子.
	\section{弱收敛}
	\quad\quad设$\mathcal{X}$是一个$B^*$空间,$\{x_n\} \subset \mathcal{X},x\in \mathcal{X}.$称$\{x_n\}$弱收敛到x,记做$x_n\rightharpoonup x$,是指:对于$\forall f \in \mathcal{X}^*$都有
	$$\lim_{n\to \infty}f(x_n) = f(x).$$
	这时x称作点列$\{x_n\}$的弱极限。\\
	强极限存在,则为弱极限。弱极限存在却未必有强极限。故命之。\\
	例 在$L^2[0,1]$中,设$x_n=x_n(t)=sinn\pi t$,则根据Riemann-Lebesgue定理,显然有
	$$<f,x_n> = \int_0^1f(t)sinn\pi t dt \to 0 (\forall f \in L^2[0,1]).$$
	即$x_n \rightharpoonup \theta$.但$\|x\|=1/\sqrt{2}$,不可能有$x_n \to 0$.\\
	对$X^*$进行上述操作得到*弱收敛。\\
	设$\mathcal{X}$是$B^*$空间,$\{f_n\}\subset\mathcal{X}^*,f\in\mathcal{X}^*$.称$f_n$*弱收敛到f,记做$\omega^*-\lim_{n\to\infty}f_n=f$,是指:对于$\forall x\in \mathcal{X},$都有$\lim_{n\to\infty}f_n(x)=f(x).$这时f称做泛函序列$\{f_n\}$的*弱极限。\\
	下面对各个收敛性进行一下归纳概括:\\
	设$\mathcal{X},\mathcal{Y}$是$B^*$空间,又设$T_n(n=1,2,\cdots),T\in\mathcal{L}(\mathcal{X},\mathcal{Y}).$
	\begin{enumerate}
		\item 若$\|T_n - T \| \to 0,$则称$T_n$一致收敛于T,记做$T_n \rightrightarrows T$.这时T称做$\{T_n\}$的一致极限。
		\item 若$\|(T_n-T)x\|\to 0(\forall x\in \mathcal{X})$,则称$T_n$强收敛于T,记做$T_n\to T$.这时T称做$\{T_n\}$的强极限。
		\item 如果对于$\forall x \in \mathcal{X}$,以及$\forall f\in \mathcal{Y}^*$,都有
			$$\lim_{n\to\infty}f(T_nx)=f(Tx),$$
			则称$T_n$弱收敛于T,记做$T_n\to T$.这时T称做$\{T_n\}$的弱极限。
	\end{enumerate}
	显然,一致收敛$\rightarrow$强收敛$\rightarrow$弱收敛,而且每种极限若存在必是唯一的。反之不然。
	\section{结束语}
	\quad\quad泛函学习至今,给我的最大感受就是转化的思想。一切都可以在空间里进行讨论,然后对其进行结构的完善,最终能够从数的角度去解释。换句话说,就是把各种元素,不管函数,还是向量,都通过范数,或次范,转移到R上研究。数学,数之学。泛函,函数的函数。范数,规范一切为数。
\end{CJK*}
\end{document}
